\documentclass[11pt,a4paper]{article}
\usepackage[top=1in, bottom=1in, left=1in, right=1in]{geometry}
\usepackage{float}
\usepackage{fancyhdr}

\usepackage{graphicx}

\usepackage{hyperref}
\hypersetup{
  colorlinks,
  citecolor=black,
  filecolor=black,
  linkcolor=black,
  urlcolor=black
}


\begin{document}

\title{\textbf{Internship Report\\Design and certification of the Mission M108}}

\author{Melanie Bombardiere}

\maketitle

\pagestyle{fancy}
\fancyhead{}
\fancyhead[LO,LE]{Design and certification process \\ of the Mission M108 LSA}

\newpage

\tableofcontents

\newpage

\section{Introduction}

During their second year at the ENSIAME (Engineering National Higher Institute in Computing, Automation, Mechanics, Energetics, and Electronics), the students must complete an internship that gives them the possibility to put into practice the knowledge they acquired in their first year.

\bigskip

This report is about the four-month internship I completed at Lambert Aircraft Engineering at the Wevelgem-Kortrijk airport in Belgium. This SME of nine employees is currently improving and developping the Mission M108, a single engine light aircraft which was the central task of the placement.

\bigskip

It will first give a brief introduction of the company and the major challenges it is facing today. It will then go on to describe the project and the team that I joined. The third section examines my contribution and tries to draw an objective analysis of the results compared to the expectations. Finally, it will describe the various things learned throughout the internship and how it matched my expectations.

\newpage

\section{Lambert Aircraft Engineering}

\subsection{Presentation of the company}

%photo of the company PIC001

The company was created by Filip and Steven LAMBERT in 1996 and is specialized in limited series production of aircrafts and in avionics maintenance. Thanks to the agreements FAA Part 145 and EASA Part M Subpart F, Lambert Aircraft Engineering is authorized to sell and install avionics products on any kind of aircrafts like Dynon, Garmin or Avidyne products.

\bigskip

The avionics workshop sells classic avionics products like headsets, GPS, transponders and so on. It sells pilot accessories, for instance lifejackets, protactors and aeronautical charts. It is also possible to design a new instrument panel for aircrafts and helicopters.

\bigskip

%TODO: diagram of the employees (Manager, 1 design engineer, 1 accountant (secretary?), 3 technicians in mechanical department, 3 in avionics department). PIC002
Decisions are taken quickly and information flows at a fast pace thanks to the simple but effective structure of the team.  There are different nationalities among the staff which demonstrates that the company is very open minded.

\bigskip

Aviation is an international environment. Lambert Aircraft Engineering does business with France, the Netherlands and the United Kingdom. The economy has been very weak since 2009 because of the economy crisis which makes things more difficult. On the aircrafts market the crisis is "noticeable", there a not many aircrafts selled anymore. But for the avionics market, it is still going very well.

\subsection{The Mission M108 and the M212}
%photo of the M108 and M212. PIC003

The M108 is a two seater single engine light sport aircraft with an unswept untapered high wing. It has a welded tubular structure and is provided with the Rotax 912iS engine. The Mission M108 can be customised to meet personal preferences for avionics and instruments. In Europe, it is available as a kit built aircraft.

\bigskip

The M212 is a four seater single engine light aircraft with an unswept untapered high wing. It has a composite airframe.

\subsection{The M108 LSA production}
The Mission M108 is a two seater single engine light sport aircraft. It is in the final stages of development and heading for certification. With the new Rotax 912iS engine installation, the design of every system is carefully reviewed and modified if required. Internal documentation for production, design and avionics departments have to be updated, such as parts lists and 2D production drawings. External documentation provided with the LSA kit to the customer in Europe, for instance the Pilot's Operating Handbook (POH), the Illustrated Parts Catalog (IPC) and the Aircraft Assembly Manual (AAM) are to be completed in parallel with the production of the first M108.

\bigskip

Employees are currently focused on the certification of the Mission M108. Technicians and engineers are working in coordination: when a part is created, the engineer has to draw it and sometimes improve it for better mechanical characteristics. There is no boundary between the production and the design teams as they are working together very closely. I was part of the design team, following the production department in order to be aware of design changes and draw or update new parts for the aicraft.

\newpage

\section{Contribution to the project}
\subsection{CAD and design work}

In order to know better about the functioning of the plane, I was assigned to draw some schematics for the Pilot's Operating Handbook. This manual explains how to maintain the aircraft in case of malfunction. This first job allowed me to get acquainted with AutoCAD and prepare myself for the future designs to come. 

%TODO: pictures of the schematics (fuel system and electrical system) PIC004 PIC005

\bigskip

I also drew the instrument panel for the Pilot's Operating Handbook to have a global view of the cabin. Depending on the LSTC chosen, we have the instrument panel with the TL instrument or the G3X system.
%TODO: picture of the instrument panel (2D) PIC006

\bigskip

I updated partslists and 2D drawings in case of design modifications, and drew some parts when it was necessary. I will take the example of parts from the radiator and the cabin heating system for explaining my contribution, but I worked on others systems in the meantime.
%TODO: picture of the radiator brackets PIC007
Designing radiator brackets seemed essential for the Mission M108 because of the original installation. Nothing from our suppliers could match our expectations. We machined the brackets ourselves and bent them for a first try. I drew and improved them in order to be a perfect fit with the baffles.
%TODO: picture of the general assembly of the radiator. PIC008

\bigskip

The cabin heating system is an optional feature. It is installed on the firewall, on the engine side. We needed two brackets to mount bowden cables to open or close the valves to heat or not the cabin. These brackets are riveted to the mounting plate and have different lengths. Since we cannot afford to reverse those two, I decided to change the dimensions of the holes for rivets on the brackets and on the mounting plate. This mistake-proofing system prevents human errors for an assembly inaccuracy.
%TODO: picture of the two brackets + picture of the assembly PIC009 PIC010

\bigskip

In parallel with the 2D drawings production, I drew 3D drawings for the Illustrated Parts Catalog (IPC). Since Lambert Aircraft Engineering sells the aircraft as a kit in Europe to homebuilders, they have an IPC in order to explain step by step how to assemble the aircraft correctly with every part referenced.
%TODO: picture of a 3D drawing with an Excel sheet PIC011 PIC012

\bigskip

I completed the sections concerning the firewall forward installation, the fuel system and the cabin heating system. I worked first on these sections because I knew how it was working thanks to the schematics. I also had to update sometimes other sections because of design modifications that occurred in the meantime.
%TODO: picture of the F4 and S4 IPC PIC013 PIC014

\bigskip

I then created all the IPC structure and the 3D drawings for the electrical, Pitot and static, and brake systems.
The electrical system is very complex because it depends on the choice of avionics products the customer wants, and the LSTCs he would like to have. I needed to find a logical structure which can group the common equipment, and separate every option to keep it as simple as possible.
%TODO: picture of the 1-S2 IPC structure PIC015

This structure takes into account the basic electrical components which are present anyway, and each option is defined with one part number. If the client needs to see how to install the navigation lights, he will need to reach the 1-S2-995300-00 drawing for further information.

%TODO: picture of the instrument panel (3D) PIC016

\bigskip

The Pitot and static systems are not new, but there was no parts list and assembly drawings for that. According with the production, I created the parts list in coordination with the previous installations.
%TODO: picture of the 1-S5 IPC PIC017

The brake system was an important part of the internship. The 2D and 3D drawings already existed but for the tailwheel version.
%TODO: difference between tailwheel and nosewheel model (schematic) PIC017 PIC018
%TODO: explain how the tailwheel version works.
In coordination with the design and the production teams, we designed the new brake system but for a nosewheel version. This was totally provisional and the design work has been done at the same time as the installation.
%TODO: picture of the 1-S6 IPC (0051-00) PIC019

We also thought that the splitter would have been installed directly in the outlet of the master brake cylinder. Indeed, the hydraulic lines are bought depending on the quantity, not the length. This design change can save one hydraulic line, approximately 300 euros.
%TODO: picture of the splitter PIC020
%TODO: picture of the previous position of the spliiter PIC021
%TODO: picture of the new position of the splitter PIC022
%TODO: picture of the 1-S6 IPC (0051-01) PIC023

The first M108 will be provided with the first version of the brakes but the next will have the splitter installed on the master brake cylinder.
%TODO: refer to 2D drawings and partslist in annexe

\subsection{Database}
LAMS (Lambert Aircraft Management System) is a new database recently created and still updated which will be used by the company. It will reference every part of the aircrafts by the part number.
%TODO: refer to the glossary section for the part number
I was assigned to test this experimental database in order to check the user interface for mistakes and improve it.

\subsubsection{Description of the database}
Status: production/provisional/withdrawn/... and definition
Type: full assembly/assembly/part bought/part made/... and definition
Reference each part by the part number without the material code (example)
Parts already ready to be ordered (example: TRIG TT21?)
Parts embedded: part which could be welded on the fuselage for example. It is available for the production but then, it is not possible to order it separately.
%TODO: structure of the database PIC024

We chose to test the database with the 1-S6 section (brake system). The parts list is quite complete and is not yet to be scheduled for any change. We also have different kinds of parts in this section and a little design change, we thus have two revisions. The brake system seemed to be the best section to have some experiments with the database.
%TODO: structure of the brake system assembly PIC025

\subsubsection{What it needs improved}
Little problems: strings too short sometimes for the description (status, material...)
Significant problems: LSTC structure: how to have a logical structure for an assembly but also with the embedded LSTC's we can have? (example of the 1-S2 section)
%TODO: structure of the 1-S2 IPC PIC026
We need to be able to find easily ALL the parts needed for a LSTC. For example, if we take the example of the LSTC 1-007 (which corresponds to the optional cabin heating system for the M108 aircraft), we should be able to know what parts we have in stock or if we need to order some of them and what parts we need to make ourselves.
This should also be available for embedded LSTC's: for instance, LSTC's 1-014 to 1-017 (radio, xpdr, com, autopilot) are only available for LSTC 1-013 (which corresponds to the TL panel).

%TODO: present the design of interior panels for cabin and my future work on that

\newpage

\section{Results}
\subsection{Evaluation of my contribution}
 I was pleased to see that even though I was an intern, I was assigned tasks that give a valuable contribution to the company. When my internship ended, all the documents and projects I had to work on were implemented and functional.
%TODO: maybe not the design project, we will see

\subsection{Technical knowledge acquired}
This was also an opportunity for me to acquire some knowledge on a large panel of technologies and methods of production.
%TODO: how an aircraft works on every aspect : mechanics (fuel, engine, cooling systems, trim control... but also the production) avionics (ECU, autopilot (pitch & roll servos), avionics shelf...)

\newpage

\section{Conclusion}
Lambert Aircraft Engineering is specialized in limited series production of aircrafts and in avionics maintenance. The objective of the internship was mainly to complete the documentation in parallel with the production of the first M108 with a new engine and in preparation of limited series production. I was part of the design team, following the production department in order to be aware of design changes and draw or update new parts for the aicraft.

\bigskip

The amount of tasks that were assigned to me, as well as their importance, was rewarding and made me feel valuable for the company. I worked on very challenging problems that deal with a large set of technologies, going from production drawings in 2D and 3D, database to design of the interior panels for the cabin. I worked closely with other engineers and technicians that were eager to assist and help me with any concern I had.

\bigskip

This internship confirmed my desire to specialize in aeronautical engineering but also developed my attraction to materials and welded structures. Working with talented, experienced engineers gave me a good understanding of what is to be expected for an engineer. The qualities that stroke me the most are: passion for aviation, willingness to leverage existing products and tools, being able to take decisions quickly, trying new technologies, and more generally getting things done!

\bigskip

I would like to thank Filip and Steven Lambert for giving me the opportunity to work on such an amazing project. I also would like to give a special thanks to Brecht Declerck for taking the time every day to help me and answering all my questions during those four months. Finally, I want to express my gratitude to all the members of the company for giving me some of their time, trusting me and letting me work in a very nice and friendly environment. It was a pleasure working with all of you.

\newpage

\section{Glossary} 
\subsection{LSA certifications}
\subsubsection{ASTM: American Society for Testing and Materials}
The Light Sport Aircraft (LSA) certification has been created by the ASTM in 2003 for aircrafts with these following characteristics:
\begin{itemize}
\item Maximal speed : 120 kt (222km/h)
\item Maximal weight: 600kg
\item Maximal stall speed : 45 kt (83km/h)
\item Single engine
\item Maximum 2 seats
\item Fixed landing gear
\item Fixed pitch or ground adjustable propeller
\item Non-pressurized cabin
\end{itemize}

\bigskip

Building aircrafts certified LSA as an amateur or professionnal is approved with the ASTM F-37 standard.

\subsubsection{EASA: European Agency for Safety Aviation}
The LSA standard from ASTM International is partially accepted by the EASA that refuses series production of light sport aircrafts in Europe. However, it allows amateurs to build one, under condition of being certified by the host country of the aircraft. This requires Lambert Aircraft Engineering to sell the Mission M108 LSA kit in Europe to homebuilders.

\subsubsection{FAA: Federal Aviation Administration}
The LSA standard from ASTM is totally accepted by the FAA and allows series and amateur production of light sport aircrafts. The core task of Lambert Aircraft Engineering is to produce M108 to the USA in series.

\subsubsection{STC's and LSTC's}
LSTC =! STC (Supplemental Type Certificate)

\subsubsection{Part classification system used in the company}
PART NUMBER (Aircraft, group, material code, part number, revision)

\end{document}
